In the last years the interest among data-intensive technologies and applications in order to extract useful information from data has increased in our day to day. These data-intensive applications and their execution environments allow users not only to handle Big Data but also to do it efficiently. This is why Big Data has become in a common technology in our lives.

Furthermore, the appearance of Big Data has generated some needs in our society. One of these needs are privacy policies. DIAs require some tools which are able to preserve the privacy of users which are generating such data. These tools can vary from international laws that legislate in this regard to technical approaches that allow to encrypt the flowing data.

Due to the development of these technologies all together and the wide range of applications where Big Data is still getting involved, writing DIA codes form scratch became in an inefficient task. At this point was where modeling techniques for software development became important for DIA development.
