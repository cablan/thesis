\chapter[Conclusions]{Conclusions}
\label{sec:conclusions}

This document faces three goals in order to develop a language which is able to apply privacy policy in streaming applications after compiling the language by means of an Acceleo compiler. Such goals are:

\begin{itemize}
\item To extend the StreamGen high level modeling approach to enable the modeling of privacy-aware streaming applications.
\item To allow software developers with few knowledge about privacy policies to handle the design of privacy-aware streaming applications.
\item From a non-privacy-aware application reach a privacy-aware application easily.
\end{itemize}

This goals are satisfied after evaluating two case studies but in spite of the privacy approach developed along this document allows to easily apply privacy policy in streaming applications by means of a metamodeling design, it takes into account that privacy policies are introduced in the DIA by means of a predefined YAML. This predefined YAML must be developed by someone with some knowledge about privacy policies and who is capable to understand the privacy requirements of the application users. This is why, it is stated as a future work to develop a metamodel which is able to generate the YAML file. Such metamodel should be filled by users in order to reduce the misunderstood that can appear when the user specifies his/her privacy requirements to a intermediary, for example a software developer. 
